\documentclass[a4paper]{report}
\usepackage{graphicx}
\usepackage{times} 
\usepackage[left=3cm,right=2.5cm,top=3cm,bottom=2.5cm]{geometry}

\begin{document}

\chapter{Pelaksanaan Kerja Praktek}

\section{Perancangan Sistem}
\subsection{Desain Antarmuka}
Desain antarmuka aplikasi manajemen faktur dirancang dengan mempertimbangkan kemudahan 
penggunaan dan efisiensi. Antarmuka utama terdiri dari beberapa komponen:

\begin{enumerate}
\item \textbf{Dashboard:} Menampilkan ringkasan statistik dan informasi penting seperti 
total faktur, status pembayaran, dan grafik tren. Dashboard ini dirancang untuk memberikan 
gambaran cepat kepada pengguna tentang kinerja bisnis mereka. Komponen ini menggunakan 
grafik interaktif yang dapat diubah ukurannya dan disesuaikan dengan kebutuhan pengguna.

\item \textbf{Manajemen Pelanggan:} Antarmuka untuk mengelola data pelanggan dengan fitur 
pencarian, penambahan, dan pengubahan data. Fitur ini memungkinkan pengguna untuk dengan 
mudah mencari pelanggan berdasarkan nama atau email, serta menambah atau mengedit 
informasi pelanggan dengan cepat.

\item \textbf{Manajemen Faktur:} Tampilan untuk membuat dan mengelola faktur dengan form 
yang intuitif dan tabel yang informatif. Formulir ini dirancang untuk meminimalkan 
kesalahan input dengan validasi real-time dan auto-complete untuk beberapa field.

\item \textbf{Pengaturan:} Panel untuk mengonfigurasi pengaturan aplikasi dan informasi 
perusahaan. Pengguna dapat mengubah informasi perusahaan, seperti nama, alamat, dan logo, 
yang akan tercermin di semua faktur yang dihasilkan.
\end{enumerate}

\subsection{Desain Database}
Aplikasi manajemen faktur ini memiliki beberapa entitas utama yang saling terhubung, yang 
dapat digambarkan dalam sebuah Entity-Relationship Diagram (ERD). Berikut adalah 
penjelasan mengenai entitas dan koneksinya:

\begin{enumerate}
\item \textbf{Users:} Tabel users menyimpan informasi pengguna yang dapat mengakses 
aplikasi. Setiap pengguna memiliki atribut seperti name, email, dan password. Tabel ini 
juga berhubungan dengan tabel sessions untuk melacak sesi pengguna dan 
password\_reset\_tokens untuk mengelola reset kata sandi.

\item \textbf{Customers:} Tabel customers menyimpan data pelanggan, termasuk nama, email, 
dan phone. Setiap pelanggan dapat memiliki banyak faktur (invoices), yang dihubungkan 
melalui customer\_id.

\item \textbf{Invoices:} Tabel invoices menyimpan informasi faktur, seperti invoice\_date, 
due\_date, email\_receiver, dan status. Setiap faktur terhubung ke satu pelanggan melalui 
customer\_id dan dapat memiliki banyak item (items), yang dihubungkan melalui invoice\_id.

\item \textbf{Items:} Tabel items menyimpan detail item yang termasuk dalam faktur, 
seperti name, description, quantity, price\_rupiah, dan price\_dollar. Setiap item 
terhubung ke satu faktur melalui invoice\_id.

\item \textbf{Exports:} Tabel exports mencatat informasi tentang ekspor data, termasuk 
file\_name, exporter, dan user\_id. Setiap ekspor terhubung ke pengguna yang melakukan 
ekspor melalui user\_id.

\item \textbf{Settings:} Tabel settings menyimpan pengaturan aplikasi, seperti 
company\_name, company\_email, dan company\_phone. Pengaturan ini digunakan di berbagai 
bagian aplikasi untuk menampilkan informasi perusahaan.

\item \textbf{Sessions:} Tabel sessions melacak sesi pengguna, menyimpan informasi seperti 
user\_id, ip\_address, dan last\_activity. Ini memungkinkan aplikasi untuk mengelola sesi 
pengguna dengan lebih baik.

\item \textbf{Password Reset Tokens:} Tabel password\_reset\_tokens menyimpan token reset 
kata sandi untuk pengguna, memungkinkan mereka untuk mengatur ulang kata sandi mereka jika 
diperlukan.
\end{enumerate}

\section{Flowchart Basis Data}
Flowchart basis data menggambarkan alur data dalam aplikasi manajemen faktur:

\begin{enumerate}
\item \textbf{Input Data Pelanggan:}
    \begin{itemize}
    \item Pengguna memasukkan data pelanggan baru
    \item Data disimpan dalam tabel customers
    \item Sistem melakukan validasi data
    \end{itemize}

\item \textbf{Pembuatan Faktur:}
    \begin{itemize}
    \item Pemilihan pelanggan dari database
    \item Input detail faktur
    \item Penambahan item-item faktur
    \item Kalkulasi otomatis total dalam IDR dan USD
    \end{itemize}

\item \textbf{Penyimpanan dan Pemrosesan:}
    \begin{itemize}
    \item Data faktur disimpan dalam tabel invoices
    \item Item-item disimpan dalam tabel items
    \item Update status pembayaran
    \end{itemize}
\end{enumerate}

\section{Sequence Diagram}
\sloppy
Sequence diagram menunjukkan interaksi antar komponen sistem dengan lebih rinci sebagai 
berikut:

\begin{enumerate}
\item \textbf{Proses Autentikasi:}
    \begin{itemize}
    \item \textbf{User melakukan login:}
        \begin{itemize}
        \item Pengguna memasukkan username dan password ke dalam form login.
        \item Form mengirimkan data kredensial ke server melalui permintaan HTTP POST.
        \end{itemize}
    \item \textbf{Sistem memverifikasi kredensial:}
        \begin{itemize}
        \item Server menerima data kredensial dan memulai proses verifikasi.
        \item Sistem mencari username di tabel \texttt{users} dalam database.
        \item Jika username ditemukan, sistem memverifikasi password yang dimasukkan 
        dengan password yang disimpan menggunakan hashing.
        \item Jika verifikasi berhasil, sistem melanjutkan ke pembuatan sesi. Jika gagal, 
        sistem mengirimkan respons kesalahan.
        \end{itemize}
    \item \textbf{Membuat sesi pengguna:}
        \begin{itemize}
        \item Sistem membuat sesi baru dengan menyimpan \texttt{user\_id}, 
        \texttt{ip\_address}, dan \texttt{last\_activity} ke tabel \texttt{sessions}.
        \item Sistem mengirimkan cookie sesi ke browser pengguna.
        \item Pengguna diarahkan ke dashboard utama aplikasi.
        \end{itemize}
    \end{itemize}

\item \textbf{Pembuatan Faktur:}
    \begin{itemize}
    \item \textbf{User memilih create invoice:}
        \begin{itemize}
        \item Pengguna mengklik tombol "Create Invoice" di dashboard.
        \item Sistem menyiapkan form pembuatan faktur.
        \end{itemize}
    \item \textbf{Sistem menampilkan form:}
        \begin{itemize}
        \item Sistem merender form dengan field untuk memilih pelanggan, memasukkan 
        tanggal faktur, dan menambahkan item.
        \item Sistem memuat data pelanggan yang tersedia untuk dipilih.
        \end{itemize}
    \item \textbf{User mengisi detail faktur:}
        \begin{itemize}
        \item Pengguna memasukkan informasi yang diperlukan, seperti memilih pelanggan, 
        menambahkan item, dan menentukan tanggal jatuh tempo.
        \item Sistem melakukan validasi real-time.
        \end{itemize}
    \item \textbf{Sistem memvalidasi dan menyimpan:}
        \begin{itemize}
        \item Sistem memeriksa semua input untuk memastikan tidak ada kesalahan.
        \item Jika validasi berhasil, sistem menyimpan data faktur ke tabel 
        \texttt{invoices} dan item ke tabel \texttt{items}.
        \item Sistem mengirimkan konfirmasi ke antarmuka pengguna.
        \end{itemize}
    \end{itemize}

\item \textbf{Pengiriman Faktur:}
    \begin{itemize}
    \item \textbf{User memilih send invoice:}
        \begin{itemize}
        \item Pengguna memilih faktur yang ingin dikirim dan mengklik opsi untuk mengirim 
        faktur.
        \end{itemize}
    \item \textbf{Sistem generate PDF:}
        \begin{itemize}
        \item Sistem menggunakan library PDF untuk merender tampilan faktur menjadi file 
        PDF.
        \item PDF disimpan sementara di server.
        \end{itemize}
    \item \textbf{Sistem mengirim email:}
        \begin{itemize}
        \item Sistem menyiapkan email dengan lampiran PDF dan detail penerima.
        \item Sistem menggunakan layanan email untuk mengirim email ke penerima.
        \item Sistem mencatat aktivitas pengiriman email dalam log.
        \end{itemize}
    \item \textbf{Update status pengiriman:}
        \begin{itemize}
        \item Sistem memeriksa apakah email berhasil dikirim.
        \item Jika berhasil, sistem memperbarui status faktur menjadi "sent" di tabel 
        \texttt{invoices}.
        \item Sistem mengirimkan notifikasi ke pengguna.
        \end{itemize}
    \end{itemize}
\end{enumerate}

\section{Activity Diagram}
Activity diagram menggambarkan alur kerja sistem:

\begin{enumerate}
\item \textbf{Login dan Autentikasi:}
    \begin{itemize}
    \item Input kredensial
    \item Verifikasi user
    \item Akses dashboard
    \end{itemize}

\item \textbf{Manajemen Faktur:}
    \begin{itemize}
    \item Pembuatan faktur baru
    \item Pengeditan faktur
    \item Pengiriman ke pelanggan
    \item Pemantauan status
    \end{itemize}

\item \textbf{Pengelolaan Data:}
    \begin{itemize}
    \item Manajemen pelanggan
    \item Update pengaturan
    \item Ekspor data
    \end{itemize}
\end{enumerate}

\section{Sistem Basis Data untuk Faktur}
Sistem basis data faktur dirancang dengan mempertimbangkan:

\begin{enumerate}
\item \textbf{Integritas Data:}
    \begin{itemize}
    \item Penggunaan foreign keys
    \item Validasi input
    \item Konsistensi data
    \end{itemize}

\item \textbf{Performa:}
    \begin{itemize}
    \item Indeks optimal
    \item Query yang efisien
    \item Caching strategis
    \end{itemize}

\item \textbf{Keamanan:}
    \begin{itemize}
    \item Enkripsi data sensitif
    \item Manajemen akses
    \item Audit trail
    \end{itemize}
\end{enumerate}

\end{document}
