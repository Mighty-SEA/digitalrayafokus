\documentclass[a4paper]{report}
\usepackage{graphicx}
\usepackage{times} 
\usepackage[left=3cm,right=2.5cm,top=3cm,bottom=2.5cm]{geometry}

% Tambahkan package untuk syntax highlighting
\usepackage{listings}
\usepackage{xcolor}

% Definisikan warna untuk syntax highlighting
\definecolor{codegreen}{rgb}{0,0.6,0}
\definecolor{codegray}{rgb}{0.5,0.5,0.5}
\definecolor{codepurple}{rgb}{0.58,0,0.82}
\definecolor{backcolour}{rgb}{0.95,0.95,0.92}

% Konfigurasi style untuk listings
\lstdefinestyle{mystyle}{
    backgroundcolor=\color{backcolour},   
    commentstyle=\color{codegreen},
    keywordstyle=\color{magenta},
    numberstyle=\tiny\color{codegray},
    stringstyle=\color{codepurple},
    basicstyle=\ttfamily\footnotesize,
    breakatwhitespace=false,         
    breaklines=true,                 
    captionpos=b,                    
    keepspaces=true,                 
    numbers=left,                    
    numbersep=5pt,                  
    showspaces=false,                
    showstringspaces=false,
    showtabs=false,                  
    tabsize=2
}

\lstset{style=mystyle}

\begin{document}

\chapter{Pelaksanaan Kerja Praktek}

\section{Perancangan Sistem}
\subsection{Desain Antarmuka}
Desain antarmuka aplikasi manajemen faktur dirancang dengan mempertimbangkan kemudahan 
penggunaan dan efisiensi. Antarmuka utama terdiri dari beberapa komponen:

\begin{enumerate}
\item \textbf{Dashboard:} Menampilkan ringkasan statistik dan informasi penting seperti 
total faktur, status pembayaran, dan grafik tren. Dashboard ini dirancang untuk memberikan 
gambaran cepat kepada pengguna tentang kinerja bisnis mereka. Komponen ini menggunakan 
grafik interaktif yang dapat diubah ukurannya dan disesuaikan dengan kebutuhan pengguna.

\item \textbf{Manajemen Pelanggan:} Antarmuka untuk mengelola data pelanggan dengan fitur 
pencarian, penambahan, dan pengubahan data. Fitur ini memungkinkan pengguna untuk dengan 
mudah mencari pelanggan berdasarkan nama atau email, serta menambah atau mengedit 
informasi pelanggan dengan cepat.

\item \textbf{Manajemen Faktur:} Tampilan untuk membuat dan mengelola faktur dengan form 
yang intuitif dan tabel yang informatif. Formulir ini dirancang untuk meminimalkan 
kesalahan input dengan validasi real-time dan auto-complete untuk beberapa field.

\item \textbf{Pengaturan:} Panel untuk mengonfigurasi pengaturan aplikasi dan informasi 
perusahaan. Pengguna dapat mengubah informasi perusahaan, seperti nama, alamat, dan logo, 
yang akan tercermin di semua faktur yang dihasilkan.
\end{enumerate}

\subsection{Desain Database}
Aplikasi manajemen faktur ini memiliki beberapa entitas utama yang saling terhubung, yang 
dapat digambarkan dalam sebuah Entity-Relationship Diagram (ERD). Berikut adalah 
penjelasan mengenai entitas dan koneksinya:

\begin{enumerate}
\item \textbf{Users:} Tabel users menyimpan informasi pengguna yang dapat mengakses 
aplikasi. Setiap pengguna memiliki atribut seperti name, email, dan password. Tabel ini 
juga berhubungan dengan tabel sessions untuk melacak sesi pengguna dan 
password\_reset\_tokens untuk mengelola reset kata sandi.

\item \textbf{Customers:} Tabel customers menyimpan data pelanggan, termasuk nama, email, 
dan phone. Setiap pelanggan dapat memiliki banyak faktur (invoices), yang dihubungkan 
melalui customer\_id.

\item \textbf{Invoices:} Tabel invoices menyimpan informasi faktur, seperti invoice\_date, 
due\_date, email\_receiver, dan status. Setiap faktur terhubung ke satu pelanggan melalui 
customer\_id dan dapat memiliki banyak item (items), yang dihubungkan melalui invoice\_id.

\item \textbf{Items:} Tabel items menyimpan detail item yang termasuk dalam faktur, 
seperti name, description, quantity, price\_rupiah, dan price\_dollar. Setiap item 
terhubung ke satu faktur melalui invoice\_id.

\item \textbf{Exports:} Tabel exports mencatat informasi tentang ekspor data, termasuk 
file\_name, exporter, dan user\_id. Setiap ekspor terhubung ke pengguna yang melakukan 
ekspor melalui user\_id.

\item \textbf{Settings:} Tabel settings menyimpan pengaturan aplikasi, seperti 
company\_name, company\_email, dan company\_phone. Pengaturan ini digunakan di berbagai 
bagian aplikasi untuk menampilkan informasi perusahaan.

\item \textbf{Sessions:} Tabel sessions melacak sesi pengguna, menyimpan informasi seperti 
user\_id, ip\_address, dan last\_activity. Ini memungkinkan aplikasi untuk mengelola sesi 
pengguna dengan lebih baik.

\item \textbf{Password Reset Tokens:} Tabel password\_reset\_tokens menyimpan token reset 
kata sandi untuk pengguna, memungkinkan mereka untuk mengatur ulang kata sandi mereka jika 
diperlukan.
\end{enumerate}

\section{Flowchart Basis Data}
Flowchart basis data menggambarkan alur data dalam aplikasi manajemen faktur:

\begin{enumerate}
\item \textbf{Input Data Pelanggan:}
    \begin{itemize}
    \item Pengguna memasukkan data pelanggan baru
    \item Data disimpan dalam tabel customers
    \item Sistem melakukan validasi data
    \end{itemize}

\item \textbf{Pembuatan Faktur:}
    \begin{itemize}
    \item Pemilihan pelanggan dari database
    \item Input detail faktur
    \item Penambahan item-item faktur
    \item Kalkulasi otomatis total dalam IDR dan USD
    \end{itemize}

\item \textbf{Penyimpanan dan Pemrosesan:}
    \begin{itemize}
    \item Data faktur disimpan dalam tabel invoices
    \item Item-item disimpan dalam tabel items
    \item Update status pembayaran
    \end{itemize}
\end{enumerate}

\section{Sequence Diagram}
\sloppy
Sequence diagram menunjukkan interaksi antar komponen sistem dengan lebih rinci sebagai 
berikut:

\begin{enumerate}
\item \textbf{Proses Autentikasi:}
    \begin{itemize}
    \item \textbf{User melakukan login:}
        \begin{itemize}
        \item Pengguna memasukkan username dan password ke dalam form login.
        \item Form mengirimkan data kredensial ke server melalui permintaan HTTP POST.
        \end{itemize}
    \item \textbf{Sistem memverifikasi kredensial:}
        \begin{itemize}
        \item Server menerima data kredensial dan memulai proses verifikasi.
        \item Sistem mencari username di tabel \texttt{users} dalam database.
        \item Jika username ditemukan, sistem memverifikasi password yang dimasukkan 
        dengan password yang disimpan menggunakan hashing.
        \item Jika verifikasi berhasil, sistem melanjutkan ke pembuatan sesi. Jika gagal, 
        sistem mengirimkan respons kesalahan.
        \end{itemize}
    \item \textbf{Membuat sesi pengguna:}
        \begin{itemize}
        \item Sistem membuat sesi baru dengan menyimpan \texttt{user\_id}, 
        \texttt{ip\_address}, dan \texttt{last\_activity} ke tabel \texttt{sessions}.
        \item Sistem mengirimkan cookie sesi ke browser pengguna.
        \item Pengguna diarahkan ke dashboard utama aplikasi.
        \end{itemize}
    \end{itemize}

\item \textbf{Pembuatan Faktur:}
    \begin{itemize}
    \item \textbf{User memilih create invoice:}
        \begin{itemize}
        \item Pengguna mengklik tombol "Create Invoice" di dashboard.
        \item Sistem menyiapkan form pembuatan faktur.
        \end{itemize}
    \item \textbf{Sistem menampilkan form:}
        \begin{itemize}
        \item Sistem merender form dengan field untuk memilih pelanggan, memasukkan 
        tanggal faktur, dan menambahkan item.
        \item Sistem memuat data pelanggan yang tersedia untuk dipilih.
        \end{itemize}
    \item \textbf{User mengisi detail faktur:}
        \begin{itemize}
        \item Pengguna memasukkan informasi yang diperlukan, seperti memilih pelanggan, 
        menambahkan item, dan menentukan tanggal jatuh tempo.
        \item Sistem melakukan validasi real-time.
        \end{itemize}
    \item \textbf{Sistem memvalidasi dan menyimpan:}
        \begin{itemize}
        \item Sistem memeriksa semua input untuk memastikan tidak ada kesalahan.
        \item Jika validasi berhasil, sistem menyimpan data faktur ke tabel 
        \texttt{invoices} dan item ke tabel \texttt{items}.
        \item Sistem mengirimkan konfirmasi ke antarmuka pengguna.
        \end{itemize}
    \end{itemize}

\item \textbf{Pengiriman Faktur:}
    \begin{itemize}
    \item \textbf{User memilih send invoice:}
        \begin{itemize}
        \item Pengguna memilih faktur yang ingin dikirim dan mengklik opsi untuk mengirim 
        faktur.
        \end{itemize}
    \item \textbf{Sistem generate PDF:}
        \begin{itemize}
        \item Sistem menggunakan library PDF untuk merender tampilan faktur menjadi file 
        PDF.
        \item PDF disimpan sementara di server.
        \end{itemize}
    \item \textbf{Sistem mengirim email:}
        \begin{itemize}
        \item Sistem menyiapkan email dengan lampiran PDF dan detail penerima.
        \item Sistem menggunakan layanan email untuk mengirim email ke penerima.
        \item Sistem mencatat aktivitas pengiriman email dalam log.
        \end{itemize}
    \item \textbf{Update status pengiriman:}
        \begin{itemize}
        \item Sistem memeriksa apakah email berhasil dikirim.
        \item Jika berhasil, sistem memperbarui status faktur menjadi "sent" di tabel 
        \texttt{invoices}.
        \item Sistem mengirimkan notifikasi ke pengguna.
        \end{itemize}
    \end{itemize}
\end{enumerate}

\section{Activity Diagram}
Activity diagram menggambarkan alur kerja sistem:

\begin{enumerate}
\item \textbf{Login dan Autentikasi:}
    \begin{itemize}
    \item Input kredensial
    \item Verifikasi user
    \item Akses dashboard
    \end{itemize}

\item \textbf{Manajemen Faktur:}
    \begin{itemize}
    \item Pembuatan faktur baru
    \item Pengeditan faktur
    \item Pengiriman ke pelanggan
    \item Pemantauan status
    \end{itemize}

\item \textbf{Pengelolaan Data:}
    \begin{itemize}
    \item Manajemen pelanggan
    \item Update pengaturan
    \item Ekspor data
    \end{itemize}
\end{enumerate}

\section{Sistem Basis Data untuk Faktur}
Sistem basis data faktur dirancang dengan mempertimbangkan:

\begin{enumerate}
\item \textbf{Integritas Data:}
    \begin{itemize}
    \item Penggunaan foreign keys
    \item Validasi input
    \item Konsistensi data
    \end{itemize}

\item \textbf{Performa:}
    \begin{itemize}
    \item Indeks optimal
    \item Query yang efisien
    \item Caching strategis
    \end{itemize}

\item \textbf{Keamanan:}
    \begin{itemize}
    \item Enkripsi data sensitif
    \item Manajemen akses
    \item Audit trail
    \end{itemize}
\end{enumerate}

\section{Implementasi Sistem}
\subsection{Implementasi Invoice}
Implementasi sistem invoice terdiri dari beberapa komponen utama:

\begin{enumerate}
\item \textbf{Resource Invoice:}
    \begin{itemize}
    \item Menggunakan framework Filament untuk membangun antarmuka CRUD
    \item Memiliki form schema yang terdiri dari dua bagian utama:
        \begin{itemize}
        \item Invoice Details: Berisi informasi dasar faktur seperti pelanggan, tanggal, dan status
        \item Items: Berisi daftar item yang dapat ditambahkan secara dinamis
        \end{itemize}
    \item Implementasi tabel dengan kolom-kolom:
        \begin{itemize}
        \item Nama pelanggan
        \item Tanggal faktur
        \item Jumlah item
        \item Total dalam IDR dan USD
        \item Status pembayaran
        \item Tanggal jatuh tempo
        \item Kurs dollar (jika menggunakan USD)
        \end{itemize}
    \end{itemize}

\item \textbf{Kalkulasi Harga:}
    \begin{itemize}
    \item Menggunakan service terpisah (InvoiceCalculationService) untuk menangani perhitungan
    \item Fitur konversi otomatis antara IDR dan USD
    \item Perhitungan total otomatis berdasarkan quantity dan harga
    \item Integrasi dengan API kurs untuk mendapatkan nilai tukar terkini
    \end{itemize}

\item \textbf{Sistem Pengiriman Email:}
    \begin{itemize}
    \item Implementasi melalui InvoiceMail class
    \item Fitur attachment PDF faktur secara otomatis
    \item Penggunaan template email yang dapat dikustomisasi
    \item Integrasi dengan pengaturan perusahaan untuk header email
    \end{itemize}

\item \textbf{Fitur Tambahan:}
    \begin{itemize}
    \item Generate PDF faktur
    \item Pengiriman faktur melalui email
    \item Manajemen status pembayaran
    \item Pencatatan riwayat perubahan
    \item Validasi input real-time
    \end{itemize}
\end{enumerate}

\subsection{Penjelasan Komponen InvoiceResource}
Berikut adalah penjelasan detail dari setiap komponen dalam file InvoiceResource.php:

\subsubsection{Deklarasi Kelas dan Properti}
Kelas InvoiceResource mewarisi kelas Resource dari Filament dengan properti utama:
\begin{itemize}
    \item \texttt{\$model}: Menentukan model yang terkait (Invoice)
    \item \texttt{\$navigationIcon}: Menentukan ikon yang ditampilkan di navigasi
\end{itemize}

\subsubsection{Fungsi Table}
Fungsi ini mendefinisikan struktur tabel untuk menampilkan data invoice dengan beberapa komponen penting:

\begin{enumerate}
\item \textbf{TextColumn:}
    \begin{itemize}
    \item \texttt{make()}: Membuat kolom dengan nama field
    \item \texttt{label()}: Label yang ditampilkan di header kolom
    \item \texttt{searchable()}: Mengaktifkan fitur pencarian
    \item \texttt{sortable()}: Mengaktifkan fitur pengurutan
    \item \texttt{weight()}: Mengatur ketebalan teks
    \item \texttt{size()}: Mengatur ukuran teks
    \item \texttt{translateLabel()}: Menerjemahkan label ke bahasa yang aktif
    \end{itemize}

\item \textbf{Formatting Kolom:}
    \begin{itemize}
    \item \texttt{formatStateUsing()}: Memformat nilai yang ditampilkan menggunakan fungsi kustom
    \item Dapat menggunakan closure function untuk format yang kompleks
    \end{itemize}

\item \textbf{Badge dan Kondisional:}
    \begin{itemize}
    \item \texttt{badge()}: Menampilkan nilai sebagai badge
    \item \texttt{color()}: Mengatur warna badge berdasarkan kondisi
    \item Mendukung pengaturan warna dinamis berdasarkan status
    \end{itemize}
\end{enumerate}

\subsubsection{Actions (Aksi Tabel)}
Sistem menyediakan beberapa aksi untuk setiap baris data:
\begin{itemize}
\item \texttt{ViewAction}: Aksi untuk melihat detail dalam modal
\item \texttt{EditAction}: Aksi untuk mengedit data
\item \texttt{modalContent()}: Mengatur konten modal
\item \texttt{icon()}: Mengatur ikon untuk aksi
\end{itemize}

\subsubsection{Form Schema}
Form schema mendefinisikan struktur formulir untuk operasi create dan edit:
\begin{itemize}
\item \texttt{Section}: Membagi form menjadi beberapa bagian
\item \texttt{schema()}: Mendefinisikan field-field dalam form
\item Mendukung validasi dan pengaturan tampilan yang fleksibel
\end{itemize}

\subsubsection{Repeater untuk Items}
Komponen untuk mengelola multiple item dalam invoice:
\begin{itemize}
\item \texttt{Repeater}: Komponen untuk menambah multiple item
\item \texttt{relationship()}: Menghubungkan dengan relasi di model
\item \texttt{live()}: Update realtime saat nilai berubah
\item Mendukung validasi per item
\end{itemize}

\subsubsection{Kalkulasi dan Helper Functions}
Sistem menyediakan fungsi pembantu untuk kalkulasi:
\begin{itemize}
\item Fungsi untuk menghitung total harga
\item \texttt{\$set}: Callback untuk mengatur nilai field
\item \texttt{\$get}: Callback untuk mendapatkan nilai field
\item Mendukung konversi mata uang otomatis
\end{itemize}

\subsubsection{Bulk Actions}
Aksi yang dapat dilakukan pada multiple baris:
\begin{itemize}
\item \texttt{DeleteBulkAction}: Aksi untuk menghapus multiple data
\item Mendukung custom bulk actions
\item Integrasi dengan sistem konfirmasi
\end{itemize}

\subsubsection{Filters dan Pengaturan Tabel}
Konfigurasi tampilan dan behavior tabel:
\begin{itemize}
\item \texttt{striped()}: Membuat tampilan tabel bergaris-garis
\item \texttt{defaultSort()}: Pengurutan default
\item \texttt{poll()}: Auto-refresh tabel setiap interval waktu
\item Mendukung pengaturan tampilan responsif
\end{itemize}

\subsubsection{Select dan Form Components}
Komponen form untuk input data:
\begin{itemize}
\item \texttt{Select}: Komponen dropdown untuk memilih data
\item \texttt{relationship()}: Menghubungkan dengan model lain
\item \texttt{createOptionForm()}: Form untuk membuat opsi baru
\item Mendukung validasi dan format input yang beragam
\end{itemize}

\subsection{Komponen Teknis}
\begin{enumerate}
\item \textbf{Form Schema:}
    \begin{itemize}
    \item Penggunaan komponen Filament seperti TextInput, DatePicker, dan Select
    \item Implementasi validasi input
    \item Fitur auto-complete untuk data pelanggan
    \item Kalkulasi otomatis untuk total dan konversi mata uang
    \end{itemize}

\item \textbf{Tabel Management:}
    \begin{itemize}
    \item Fitur pencarian dan pengurutan
    \item Filter berdasarkan status dan tanggal
    \item Aksi bulk untuk multiple invoice
    \item Tampilan responsif dan mudah digunakan
    \end{itemize}

\item \textbf{Integrasi API:}
    \begin{itemize}
    \item Koneksi dengan API kurs mata uang
    \item Pembaruan otomatis nilai tukar
    \item Penyimpanan kurs terakhir di pengaturan sistem
    \end{itemize}
\end{enumerate}

\subsection{Implementasi Komponen Sistem}

\subsubsection{Model Relationships dan Database}
Model Invoice memiliki beberapa relasi dan konfigurasi penting:
\begin{itemize}
\item \texttt{\$guarded = []}: Mengizinkan mass assignment untuk semua field
\item \texttt{\$dates}: Mendefinisikan kolom yang akan dikonversi ke Carbon instances
\item Relasi:
    \begin{itemize}
    \item \texttt{hasMany} dengan Item: Satu invoice bisa memiliki banyak item
    \item \texttt{belongsTo} dengan Customer: Satu invoice dimiliki oleh satu customer
    \end{itemize}
\end{itemize}

\subsubsection{Service Layer untuk Kalkulasi}
Service khusus untuk kalkulasi invoice dengan fungsi-fungsi:
\begin{itemize}
\item \texttt{calculateAmounts}: Menghitung total berdasarkan quantity
\item \texttt{convertCurrency}: Konversi mata uang IDR-USD
\item Menggunakan \texttt{round()} untuk presisi angka
\end{itemize}

\subsubsection{Email Notification System}
Sistem notifikasi email menggunakan kelas \texttt{InvoiceMail}:
\begin{itemize}
\item Menggunakan template blade
\item Menyertakan data perusahaan dari settings
\item Attachment PDF invoice otomatis
\item Konfigurasi yang fleksibel untuk konten email
\end{itemize}

\subsubsection{Statistik Dashboard}
Widget statistik untuk monitoring invoice:
\begin{itemize}
\item Menghitung total pesanan
\item Persentase pembatalan
\item Rasio pembayaran
\item Auto-update setiap interval waktu
\end{itemize}

\subsubsection{Bulk Actions}
Aksi massal untuk multiple invoice:
\begin{itemize}
\item Generate PDF batch
\item Kirim email batch
\item Menggunakan collection untuk efisiensi
\item Validasi batch operation
\end{itemize}

\subsubsection{Database Migrations}
Struktur tabel untuk sistem invoice:
\begin{itemize}
\item Foreign key ke invoice
\item Support dual currency (IDR dan USD)
\item Soft delete untuk data safety
\item Indeks untuk optimasi query
\end{itemize}

\subsubsection{Export Functionality}
Fitur ekspor data invoice:
\begin{itemize}
\item Export ke format Excel
\item Kustomisasi format kolom
\item Kalkulasi total otomatis
\item Support formatting angka
\end{itemize}

\subsubsection{View Components}
Komponen blade untuk tampilan:
\begin{itemize}
\item Status badge dengan styling dinamis
\item Reusable di berbagai view
\item Support tema warna yang konsisten
\item Komponen modular
\end{itemize}

\subsubsection{Sistem Notifikasi}
Multi-channel notification system:
\begin{itemize}
\item Notifikasi database dan email
\item Format pesan yang customizable
\item Tracking status notifikasi
\item Data terstruktur untuk database
\end{itemize}

\subsubsection{Validasi dan Request Rules}
Sistem validasi komprehensif:
\begin{itemize}
\item Validasi form yang menyeluruh
\item Custom error messages
\item Validasi nested array untuk items
\item Conditional validation untuk currency
\end{itemize}

\subsubsection{Event dan Listeners}
Sistem event untuk invoice lifecycle:
\begin{itemize}
\item Queue system untuk background processing
\item Activity logging
\item Cache management
\item Event handling yang terstruktur
\end{itemize}

\subsubsection{PDF Generation Service}
Service khusus untuk generate PDF:
\begin{itemize}
\item Integrasi QR code
\item Custom paper size dan format
\item File management system
\item Template yang fleksibel
\end{itemize}

\subsubsection{Currency Conversion Service}
Service untuk konversi mata uang:
\begin{itemize}
\item API integration untuk kurs
\item Caching untuk optimasi
\item Error handling
\item Presisi perhitungan
\end{itemize}

\subsubsection{Report Generation}
Sistem pembuatan laporan:
\begin{itemize}
\item Laporan bulanan invoice
\item Breakdown berdasarkan status
\item Analisis per customer
\item Kalkulasi multi-currency
\end{itemize}

\subsubsection{Command Line Interface}
CLI untuk manajemen sistem:
\begin{itemize}
\item Generate report otomatis
\item Multiple format output
\item Parameter yang fleksibel
\item Feedback yang informatif
\end{itemize}

\subsection{Implementasi Kode}
Berikut adalah implementasi detail dari beberapa komponen utama:

\subsubsection{Model Invoice}
\begin{lstlisting}[language=PHP]
class Invoice extends Model
{
    protected $guarded = [];
    use HasFactory;

    protected $dates = [
        'invoice_date',
        'due_date',
        'created_at',
        'updated_at'
    ];

    public function Item(){
        return $this->hasMany(Item::class);
    }

    public function Customer(){
        return $this->belongsTo(Customer::class);
    }
}
\end{lstlisting}

\subsubsection{Service Kalkulasi}
\begin{lstlisting}[language=PHP]
class InvoiceCalculationService 
{
    public static function calculateAmounts(
        float $quantity, 
        float $priceRupiah, 
        float $priceDollar
    ): array 
    {
        return [
            'amount_rupiah' => $priceRupiah * $quantity,
            'amount_dollar' => $priceDollar * $quantity
        ];
    }

    public static function convertCurrency(
        float $price, 
        float $currentDollar, 
        string $type
    ): array 
    {
        if ($type === 'rupiah') {
            $priceDollar = round($price / $currentDollar, 2);
            return [
                'price_rupiah' => $price,
                'price_dollar' => $priceDollar
            ];
        }

        $priceRupiah = round($price * $currentDollar);
        return [
            'price_rupiah' => $priceRupiah, 
            'price_dollar' => $price
        ];
    }
}
\end{lstlisting}

\subsubsection{Notifikasi Email}
\begin{lstlisting}[language=PHP]
class InvoiceMail extends Mailable
{
    public function content(): Content
    {
        return new Content(
            view: "invoices.send",
            with: [
                "invoice" => $this->invoice,
                "settings" => [
                    'name' => Settings::get('company_name'),
                    'email' => Settings::get('company_email'),
                    'phone' => Settings::get('company_phone'),
                    'address' => Settings::get('company_address'),
                    'logo' => Settings::get('company_logo'),
                ]
            ]
        );
    }

    public function attachments(): array
    {
        return [
            Attachment::fromPath(
                storage_path(
                    "app/invoices/" .
                        $this->invoice->id .
                        "-" .
                        $this->invoice->customer->nama .
                        ".pdf"
                )
            ),
        ];
    }
}
\end{lstlisting}

\subsubsection{PDF Generation Implementation}
\begin{lstlisting}[language=PHP]
class PdfGenerationService
{
    public function generateInvoicePdf(Invoice $invoice): string
    {
        $pdf = PDF::loadView('invoices.pdf', [
            'invoice' => $invoice,
            'settings' => $this->getCompanySettings(),
            'qrCode' => $this->generateQrCode($invoice),
        ]);

        $pdf->setPaper('a4');
        
        $filename = sprintf(
            'invoice_%s_%s.pdf',
            $invoice->id,
            now()->format('Y-m-d')
        );
        
        $path = storage_path("app/invoices/{$filename}");
        $pdf->save($path);
        
        return $path;
    }

    private function generateQrCode(Invoice $invoice): string
    {
        return QrCode::format('svg')
            ->size(100)
            ->generate($invoice->payment_url);
    }
}
\end{lstlisting}

\subsubsection{Report Generation Implementation}
\begin{lstlisting}[language=PHP]
class InvoiceReport
{
    public function generateMonthlyReport(Carbon $month): array
    {
        $invoices = Invoice::whereMonth('created_at', $month)
            ->with(['customer', 'items'])
            ->get();

        return [
            'total_invoices' => $invoices->count(),
            'total_amount' => [
                'idr' => $invoices->sum(function ($invoice) {
                    return $invoice->items->sum('amount_rupiah');
                }),
                'usd' => $invoices->sum(function ($invoice) {
                    return $invoice->items->sum('amount_dollar');
                }),
            ],
            'status_breakdown' => $invoices->groupBy('status')
                ->map(fn ($group) => $group->count()),
            'customer_breakdown' => $invoices->groupBy('customer_id')
                ->map(function ($group) {
                    return [
                        'count' => $group->count(),
                        'total' => $group->sum(function ($invoice) {
                            return $invoice->items->sum('amount_rupiah');
                        }),
                    ];
                }),
        ];
    }
}
\end{lstlisting}

\subsubsection{Currency Service Implementation}
\begin{lstlisting}[language=PHP]
class CurrencyService
{
    private $api_key;
    private $cache_duration = 3600; // 1 jam

    public function __construct()
    {
        $this->api_key = config('services.currency.api_key');
    }

    public function getLatestRate(): float
    {
        return Cache::remember('usd_rate', $this->cache_duration, function () {
            $response = Http::get("https://api.exchangerate-api.com/v4/latest/USD", [
                'apikey' => $this->api_key
            ]);

            if ($response->successful()) {
                return $response->json()['rates']['IDR'];
            }

            throw new CurrencyServiceException('Failed to fetch exchange rate');
        });
    }

    public function convertToRupiah(float $usdAmount): float
    {
        $rate = $this->getLatestRate();
        return round($usdAmount * $rate, 2);
    }
}
\end{lstlisting}

\subsubsection{Command Line Interface Implementation}
\begin{lstlisting}[language=PHP]
class GenerateInvoiceReport extends Command
{
    protected $signature = 'invoice:report 
        {month? : Bulan dalam format YYYY-MM} 
        {--format=pdf : Format output (pdf/excel/csv)}';

    protected $description = 'Generate laporan invoice bulanan';

    public function handle(InvoiceReport $reporter)
    {
        $month = Carbon::parse($this->argument('month') ?? now());
        $format = $this->option('format');

        $report = $reporter->generateMonthlyReport($month);

        match ($format) {
            'pdf' => $this->generatePdf($report),
            'excel' => $this->generateExcel($report),
            'csv' => $this->generateCsv($report),
            default => $this->error('Format tidak didukung'),
        };

        $this->info('Laporan berhasil dibuat!');
    }
}
\end{lstlisting}

Penjelasan komponen:
\begin{itemize}
\item \texttt{protected \$signature}:
    \begin{itemize}
    \item Mendefinisikan format perintah CLI
    \item Parameter opsional untuk bulan
    \item Option untuk format output
    \item Syntax yang mudah dibaca
    \end{itemize}

\item \texttt{protected \$description}:
    \begin{itemize}
    \item Deskripsi perintah untuk dokumentasi
    \item Muncul saat menjalankan \texttt{php artisan list}
    \item Membantu pengguna memahami fungsi command
    \end{itemize}

\item \texttt{handle()}:
    \begin{itemize}
    \item Method utama yang dieksekusi
    \item Dependency injection untuk reporter
    \item Parsing parameter bulan
    \item Pemilihan format output
    \end{itemize}

\item \texttt{match expression}:
    \begin{itemize}
    \item Switch case modern PHP 8+
    \item Penanganan berbagai format output
    \item Error handling untuk format tidak valid
    \item Return value yang konsisten
    \end{itemize}
\end{itemize}

\subsubsection{Notification System Implementation}
\begin{lstlisting}[language=PHP]
class InvoiceNotification extends Notification implements ShouldQueue
{
    public function __construct(private Invoice $invoice)
    {
        $this->afterCommit = true;
    }

    public function via($notifiable): array
    {
        return ['mail', 'database'];
    }

    public function toMail($notifiable): MailMessage
    {
        return (new MailMessage)
            ->subject('Invoice #' . $this->invoice->id)
            ->markdown('mail.invoice.created', [
                'invoice' => $this->invoice,
                'url' => URL::signedRoute('invoice.show', [
                    'invoice' => $this->invoice->id
                ])
            ]);
    }

    public function toDatabase($notifiable): array
    {
        return [
            'invoice_id' => $this->invoice->id,
            'amount' => $this->invoice->total_amount,
            'customer' => $this->invoice->customer->nama,
            'status' => $this->invoice->status,
        ];
    }
}
\end{lstlisting}

Penjelasan komponen:
\begin{itemize}
\item \texttt{ShouldQueue interface}:
    \begin{itemize}
    \item Implementasi antrian untuk notifikasi
    \item Proses asynchronous
    \item Peningkatan performa aplikasi
    \item Retry mechanism untuk kegagalan
    \end{itemize}

\item \texttt{via()}:
    \begin{itemize}
    \item Menentukan channel notifikasi
    \item Support multiple channels
    \item Fleksibilitas pengiriman
    \item Mudah dikonfigurasi
    \end{itemize}

\item \texttt{toMail()}:
    \begin{itemize}
    \item Konfigurasi email notification
    \item Menggunakan markdown template
    \item Signed URL untuk keamanan
    \item Customizable content
    \end{itemize}

\item \texttt{toDatabase()}:
    \begin{itemize}
    \item Format data untuk penyimpanan di database
    \item Struktur data yang konsisten
    \item Tracking notifikasi
    \item Query-friendly format
    \end{itemize}
\end{itemize}

\subsubsection{Export Service Implementation}
\begin{lstlisting}[language=PHP]
class InvoiceExportService
{
    public function exportToExcel(Collection $invoices, string $filename): string
    {
        return Excel::download(
            new InvoicesExport($invoices),
            $filename
        )->getFile()->getPathname();
    }

    public function exportToCsv(Collection $invoices, string $filename): string
    {
        return Excel::download(
            new InvoicesExport($invoices),
            $filename,
            \Maatwebsite\Excel\Excel::CSV
        )->getFile()->getPathname();
    }

    private function getExportColumns(): array
    {
        return [
            'ID' => 'id',
            'Customer' => 'customer.nama',
            'Date' => 'invoice_date',
            'Due Date' => 'due_date',
            'Status' => 'status',
            'Total (IDR)' => 'total_amount_idr',
            'Total (USD)' => 'total_amount_usd',
        ];
    }
}
\end{lstlisting}

Penjelasan komponen:
\begin{itemize}
\item \texttt{exportToExcel()}:
    \begin{itemize}
    \item Ekspor data ke format Excel
    \item Parameter collection untuk batch processing
    \item Kustomisasi nama file
    \item Return path file hasil ekspor
    \end{itemize}

\item \texttt{exportToCsv()}:
    \begin{itemize}
    \item Ekspor data ke format CSV
    \item Menggunakan library Maatwebsite Excel
    \item Support karakter encoding
    \item Optimasi untuk file besar
    \end{itemize}

\item \texttt{getExportColumns()}:
    \begin{itemize}
    \item Definisi kolom untuk ekspor
    \item Mapping field database ke header
    \item Support relasi database
    \item Format yang konsisten
    \end{itemize}
\end{itemize}

\subsection{Optimasi dan Performa}
Beberapa teknik optimasi yang diimplementasikan dalam sistem:

\begin{enumerate}
\item \textbf{Database Query Optimization:}
    \begin{itemize}
    \item Penggunaan eager loading untuk menghindari N+1 query problem
    \item Indexing pada kolom-kolom yang sering digunakan dalam pencarian
    \item Query caching untuk data yang jarang berubah
    \item Chunk processing untuk data dalam jumlah besar
    \end{itemize}

\item \textbf{Cache Implementation:}
    \begin{itemize}
    \item Caching kurs mata uang
    \item Cache template PDF yang sering digunakan
    \item Cache query database yang kompleks
    \item Automatic cache invalidation
    \end{itemize}

\item \textbf{Background Processing:}
    \begin{itemize}
    \item Queue untuk pengiriman email
    \item Async PDF generation
    \item Scheduled report generation
    \item Rate limiting untuk API calls
    \end{itemize}

\item \textbf{Resource Management:}
    \begin{itemize}
    \item Automatic cleanup untuk file temporary
    \item Optimasi ukuran gambar
    \item Memory usage monitoring
    \item Connection pooling
    \end{itemize}
\end{enumerate}

\subsection{Security Measures}
Implementasi keamanan sistem meliputi:

\begin{enumerate}
\item \textbf{Authentication:}
    \begin{itemize}
    \item Multi-factor authentication
    \item Session management
    \item Password policy enforcement
    \item Login attempt limitation
    \end{itemize}

\item \textbf{Data Protection:}
    \begin{itemize}
    \item Encryption at rest
    \item Secure file storage
    \item Data backup routines
    \item Access control lists
    \end{itemize}

\item \textbf{API Security:}
    \begin{itemize}
    \item Token-based authentication
    \item Request validation
    \item Rate limiting
    \item CORS policy
    \end{itemize}

\item \textbf{Audit Trail:}
    \begin{itemize}
    \item Activity logging
    \item Change tracking
    \item User action history
    \item System event monitoring
    \end{itemize}
\end{enumerate}

\subsection{Implementasi Kode Detail}

\subsubsection{Model Invoice}
\begin{lstlisting}[language=PHP]
class Invoice extends Model
{
    protected $guarded = [];
    use HasFactory;

    protected $dates = [
        'invoice_date',
        'due_date',
        'created_at',
        'updated_at'
    ];

    public function Item(){
        return $this->hasMany(Item::class);
    }

    public function Customer(){
        return $this->belongsTo(Customer::class);
    }
}
\end{lstlisting}

Penjelasan komponen:
\begin{itemize}
\item \texttt{protected \$guarded = []}: 
    \begin{itemize}
    \item Mengizinkan mass assignment untuk semua atribut
    \item Digunakan untuk operasi create dan update data
    \item Perlu berhati-hati dengan keamanan input
    \end{itemize}

\item \texttt{protected \$dates}:
    \begin{itemize}
    \item Mendefinisikan kolom yang akan dikonversi ke Carbon
    \item Memungkinkan manipulasi tanggal yang lebih mudah
    \item Otomatis mengkonversi format database ke format PHP
    \end{itemize}

\item \texttt{public function Item()}:
    \begin{itemize}
    \item Relasi one-to-many dengan model Item
    \item Memungkinkan akses \texttt{\$invoice->items}
    \item Mendukung eager loading untuk optimasi query
    \end{itemize}

\item \texttt{public function Customer()}:
    \begin{itemize}
    \item Relasi many-to-one dengan model Customer
    \item Memungkinkan akses \texttt{\$invoice->customer}
    \item Foreign key constraint untuk integritas data
    \end{itemize}
\end{itemize}

\subsubsection{PDF Generation Service}
\begin{lstlisting}[language=PHP]
class PdfGenerationService
{
    public function generateInvoicePdf(Invoice $invoice): string
    {
        $pdf = PDF::loadView('invoices.pdf', [
            'invoice' => $invoice,
            'settings' => $this->getCompanySettings(),
            'qrCode' => $this->generateQrCode($invoice),
        ]);

        $pdf->setPaper('a4');
        
        $filename = sprintf(
            'invoice_%s_%s.pdf',
            $invoice->id,
            now()->format('Y-m-d')
        );
        
        $path = storage_path("app/invoices/{$filename}");
        $pdf->save($path);
        
        return $path;
    }

    private function generateQrCode(Invoice $invoice): string
    {
        return QrCode::format('svg')
            ->size(100)
            ->generate($invoice->payment_url);
    }
}
\end{lstlisting}

Penjelasan komponen:
\begin{itemize}
\item \texttt{generateInvoicePdf()}:
    \begin{itemize}
    \item Parameter: objek Invoice yang akan di-generate
    \item Return: string path file PDF yang dihasilkan
    \item Menggunakan view blade untuk template PDF
    \item Menyertakan pengaturan perusahaan dan QR code
    \end{itemize}

\item \texttt{PDF::loadView()}:
    \begin{itemize}
    \item Memuat template blade untuk PDF
    \item Menerima data yang akan dirender
    \item Menggunakan library DomPDF
    \item Mendukung styling CSS
    \end{itemize}

\item \texttt{setPaper()}:
    \begin{itemize}
    \item Mengatur ukuran kertas PDF
    \item Default menggunakan A4
    \item Mendukung orientasi portrait/landscape
    \end{itemize}

\item \texttt{generateQrCode()}:
    \begin{itemize}
    \item Membuat QR code untuk payment URL
    \item Format SVG untuk kualitas tinggi
    \item Ukuran 100x100 pixel
    \item Mendukung error correction
    \end{itemize}
\end{itemize}

\subsubsection{Currency Service}
\begin{lstlisting}[language=PHP]
class CurrencyService
{
    private $api_key;
    private $cache_duration = 3600; // 1 jam

    public function __construct()
    {
        $this->api_key = config('services.currency.api_key');
    }

    public function getLatestRate(): float
    {
        return Cache::remember('usd_rate', $this->cache_duration, function () {
            $response = Http::get("https://api.exchangerate-api.com/v4/latest/USD", [
                'apikey' => $this->api_key
            ]);

            if ($response->successful()) {
                return $response->json()['rates']['IDR'];
            }

            throw new CurrencyServiceException('Failed to fetch exchange rate');
        });
    }

    public function convertToRupiah(float $usdAmount): float
    {
        $rate = $this->getLatestRate();
        return round($usdAmount * $rate, 2);
    }
}
\end{lstlisting}

Penjelasan komponen:
\begin{itemize}
\item \texttt{private \$api\_key}:
    \begin{itemize}
    \item Menyimpan API key untuk layanan kurs
    \item Diambil dari konfigurasi aplikasi
    \item Aman dari eksposur publik
    \end{itemize}

\item \texttt{private \$cache\_duration}:
    \begin{itemize}
    \item Durasi cache kurs (1 jam)
    \item Mengurangi request ke API
    \item Optimasi performa
    \end{itemize}

\item \texttt{getLatestRate()}:
    \begin{itemize}
    \item Mengambil kurs USD-IDR terbaru
    \item Menggunakan sistem cache
    \item Error handling untuk kegagalan API
    \item Return nilai float kurs
    \end{itemize}

\item \texttt{convertToRupiah()}:
    \begin{itemize}
    \item Konversi USD ke IDR
    \item Menggunakan kurs terbaru
    \item Pembulatan 2 desimal
    \item Return nilai dalam Rupiah
    \end{itemize}
\end{itemize}

\subsubsection{Report Generation}
\begin{lstlisting}[language=PHP]
class InvoiceReport
{
    public function generateMonthlyReport(Carbon $month): array
    {
        $invoices = Invoice::whereMonth('created_at', $month)
            ->with(['customer', 'items'])
            ->get();

        return [
            'total_invoices' => $invoices->count(),
            'total_amount' => [
                'idr' => $invoices->sum(function ($invoice) {
                    return $invoice->items->sum('amount_rupiah');
                }),
                'usd' => $invoices->sum(function ($invoice) {
                    return $invoice->items->sum('amount_dollar');
                }),
            ],
            'status_breakdown' => $invoices->groupBy('status')
                ->map(fn ($group) => $group->count()),
            'customer_breakdown' => $invoices->groupBy('customer_id')
                ->map(function ($group) {
                    return [
                        'count' => $group->count(),
                        'total' => $group->sum(function ($invoice) {
                            return $invoice->items->sum('amount_rupiah');
                        }),
                    ];
                }),
        ];
    }
}
\end{lstlisting}

Penjelasan komponen:
\begin{itemize}
\item \texttt{generateMonthlyReport()}:
    \begin{itemize}
    \item Parameter: objek Carbon untuk bulan yang diinginkan
    \item Return: array berisi statistik laporan
    \item Menggunakan eager loading untuk optimasi
    \item Menghitung berbagai metrik
    \end{itemize}

\item \texttt{whereMonth()}:
    \begin{itemize}
    \item Filter invoice berdasarkan bulan
    \item Menggunakan kolom created\_at
    \item Query builder Laravel
    \end{itemize}

\item \texttt{with()}:
    \begin{itemize}
    \item Eager loading relasi
    \item Menghindari N+1 query problem
    \item Meningkatkan performa
    \end{itemize}

\item \texttt{groupBy() dan map()}:
    \begin{itemize}
    \item Mengelompokkan data berdasarkan kriteria
    \item Transformasi data untuk laporan
    \item Kalkulasi statistik per kelompok
    \end{itemize}
\end{itemize}

\subsubsection{Command Line Interface Implementation}
\begin{lstlisting}[language=PHP]
class GenerateInvoiceReport extends Command
{
    protected $signature = 'invoice:report 
        {month? : Bulan dalam format YYYY-MM} 
        {--format=pdf : Format output (pdf/excel/csv)}';

    protected $description = 'Generate laporan invoice bulanan';

    public function handle(InvoiceReport $reporter)
    {
        $month = Carbon::parse($this->argument('month') ?? now());
        $format = $this->option('format');

        $report = $reporter->generateMonthlyReport($month);

        match ($format) {
            'pdf' => $this->generatePdf($report),
            'excel' => $this->generateExcel($report),
            'csv' => $this->generateCsv($report),
            default => $this->error('Format tidak didukung'),
        };

        $this->info('Laporan berhasil dibuat!');
    }
}
\end{lstlisting}

Penjelasan komponen:
\begin{itemize}
\item \texttt{protected \$signature}:
    \begin{itemize}
    \item Mendefinisikan format perintah CLI
    \item Parameter opsional untuk bulan
    \item Option untuk format output
    \item Syntax yang mudah dibaca
    \end{itemize}

\item \texttt{protected \$description}:
    \begin{itemize}
    \item Deskripsi perintah untuk dokumentasi
    \item Muncul saat menjalankan \texttt{php artisan list}
    \item Membantu pengguna memahami fungsi command
    \end{itemize}

\item \texttt{handle()}:
    \begin{itemize}
    \item Method utama yang dieksekusi
    \item Dependency injection untuk reporter
    \item Parsing parameter bulan
    \item Pemilihan format output
    \end{itemize}

\item \texttt{match expression}:
    \begin{itemize}
    \item Switch case modern PHP 8+
    \item Penanganan berbagai format output
    \item Error handling untuk format tidak valid
    \item Return value yang konsisten
    \end{itemize}
\end{itemize}

\subsubsection{Notification System Implementation}
\begin{lstlisting}[language=PHP]
class InvoiceNotification extends Notification implements ShouldQueue
{
    public function __construct(private Invoice $invoice)
    {
        $this->afterCommit = true;
    }

    public function via($notifiable): array
    {
        return ['mail', 'database'];
    }

    public function toMail($notifiable): MailMessage
    {
        return (new MailMessage)
            ->subject('Invoice #' . $this->invoice->id)
            ->markdown('mail.invoice.created', [
                'invoice' => $this->invoice,
                'url' => URL::signedRoute('invoice.show', [
                    'invoice' => $this->invoice->id
                ])
            ]);
    }

    public function toDatabase($notifiable): array
    {
        return [
            'invoice_id' => $this->invoice->id,
            'amount' => $this->invoice->total_amount,
            'customer' => $this->invoice->customer->nama,
            'status' => $this->invoice->status,
        ];
    }
}
\end{lstlisting}

Penjelasan komponen:
\begin{itemize}
\item \texttt{ShouldQueue interface}:
    \begin{itemize}
    \item Implementasi antrian untuk notifikasi
    \item Proses asynchronous
    \item Peningkatan performa aplikasi
    \item Retry mechanism untuk kegagalan
    \end{itemize}

\item \texttt{via()}:
    \begin{itemize}
    \item Menentukan channel notifikasi
    \item Support multiple channels
    \item Fleksibilitas pengiriman
    \item Mudah dikonfigurasi
    \end{itemize}

\item \texttt{toMail()}:
    \begin{itemize}
    \item Konfigurasi email notification
    \item Menggunakan markdown template
    \item Signed URL untuk keamanan
    \item Customizable content
    \end{itemize}

\item \texttt{toDatabase()}:
    \begin{itemize}
    \item Format data untuk penyimpanan di database
    \item Struktur data yang konsisten
    \item Tracking notifikasi
    \item Query-friendly format
    \end{itemize}
\end{itemize}

\subsubsection{Export Service Implementation}
\begin{lstlisting}[language=PHP]
class InvoiceExportService
{
    public function exportToExcel(Collection $invoices, string $filename): string
    {
        return Excel::download(
            new InvoicesExport($invoices),
            $filename
        )->getFile()->getPathname();
    }

    public function exportToCsv(Collection $invoices, string $filename): string
    {
        return Excel::download(
            new InvoicesExport($invoices),
            $filename,
            \Maatwebsite\Excel\Excel::CSV
        )->getFile()->getPathname();
    }

    private function getExportColumns(): array
    {
        return [
            'ID' => 'id',
            'Customer' => 'customer.nama',
            'Date' => 'invoice_date',
            'Due Date' => 'due_date',
            'Status' => 'status',
            'Total (IDR)' => 'total_amount_idr',
            'Total (USD)' => 'total_amount_usd',
        ];
    }
}
\end{lstlisting}

Penjelasan komponen:
\begin{itemize}
\item \texttt{exportToExcel()}:
    \begin{itemize}
    \item Ekspor data ke format Excel
    \item Parameter collection untuk batch processing
    \item Kustomisasi nama file
    \item Return path file hasil ekspor
    \end{itemize}

\item \texttt{exportToCsv()}:
    \begin{itemize}
    \item Ekspor data ke format CSV
    \item Menggunakan library Maatwebsite Excel
    \item Support karakter encoding
    \item Optimasi untuk file besar
    \end{itemize}

\item \texttt{getExportColumns()}:
    \begin{itemize}
    \item Definisi kolom untuk ekspor
    \item Mapping field database ke header
    \item Support relasi database
    \item Format yang konsisten
    \end{itemize}
\end{itemize}

\end{document}
